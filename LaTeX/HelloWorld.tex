\documentclass{article}
\usepackage{geometry}
\usepackage{fancyhdr}
\usepackage{amsmath, amsthm, amssymb}
\usepackage{graphicx}
\usepackage{hyperref}
\usepackage{lipsum}
%\usepackage[utf8]{inputenc}

\title{Test document}
\author{Aleksander V\aa ge \\ \url{Alek@vage.com}}
\date{13. March 2018}

\begin{document}

\pagestyle{fancy}
\lhead{\today}
\chead{}
\rhead{Test Document}
\lfoot{}
\cfoot{}
\rfoot{\thepage}

\renewcommand{\familydefault}{\sfdefault}
\fontfamily{cmss}\selectfont

\maketitle
\tableofcontents
\newpage

Hello World! \LaTeX

This is where I can write some preamble text. \footnote{First footnote.}\footnote{Second footnote.}

\section{Text for the first section}
\lipsum[1]

\subsection{Text for a subsection of the first section}
\lipsum[2-3]
\label{labelone}

\subsection{Another subsection of the first section, not a subsubsection}
\lipsum[4-5]
\label{labeltwo}

\section{The second section}
\lipsum[6]

Refer again to \ref{labelone}.
Note also the discussion on page \pageref{labeltwo}

\subsection{Title of the first subsection of the second section}
\lipsum[7]


\section{The mathematics section}
There are $\binom{2n+1}{n}$ sequences with $n$ occurrences of $-1$ and $n+1$ occurrences of $+1$, and Raneys's lemma tells us that exactly $1/(2n+1)$ of these sequences have all partial sums positive.

\subsection{Elementary calculus}
Elementary cacluclus suffices to evaluate $C$ if we are clever enough to look at the double integral
\begin{equation*}
  C^2
  = \int_{-\infty}^{+\infty} e^{-x^2} \mathrm{d}x
  \int_{-\infty}^{+\infty} e^{-y^2} \mathrm{d}y\;.
\end{equation*}

\subsubsection{Task 1}
Solve the following recurrence for $n, k\geq 0$:
\begin{align*}
  Q_{n,0} &= 1
  \quad Q_{0,k} = [k=0]; \\
  q_{n,k} &= Q_{n-1,k}+Q_{n-1,k-1}+\binom{n}{k}, \quad\text{for $n, k>0$.}
\end{align*}

\subsubsection{Examples of AMS Symbols}
Therefore
\begin{equation*}
  a\equiv b\pmod{m}
  \qquad\Longleftrightarrow\qquad
  a\equiv b \pmod{p^{m_p}}\quad\text{for all $p$}
\end{equation*}
if the prime factorization of $m$ is $\prod_p p^{m_p}$

\begin{thebibliography}{9}
  
\bibitem{ConcreteMath}
  Ronald L. Graham, Donald E. Knuth, and Oren Patashnik,
  \textit{Concrete Mathematics},
  Addison-Wesley, Reading, MA, 1995.
  
\end{thebibliograhy}

\end{document}
